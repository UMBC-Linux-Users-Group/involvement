\documentclass[11pt]{article}

\usepackage[margin=0.25in,landscape]{geometry}
\usepackage{multicol}
\usepackage{textcomp} % for copyleft symbol
\usepackage{lipsum}   % for dummy text
\usepackage{tabulary} % for fixed-width tables

\newcommand{\foss}{FOSS}
\newcommand{\tbd}{TBD}

\setlength{\columnsep}{0.5in}
\def\columnseprule{0.5pt}

\begin{document}
\begin{multicols*}{2}
%%%%%%%% PAGE 1 %%%%%%%%

\begin{center} \scshape \Huge
    \phantom{x}
    Linux Users' Group
    %\vspace{0.25in}
\end{center}

\begin{center} \Large
    Meetings
    Every Wednesday, Noon to 1 p.m. in ITE 234
\end{center}

\section*{Who We Are}

We are students who use or want to use Linux. We meet weekly to chat and help
each other solve problems, and organize other events to promote free and open
source software.

% May want to re-order these, though ``What is linux'' is probably a good thing to open
% with.
\section*{What Is Linux}

Linux is an operating system similar to OS X or Windows, except that it is free
and open source, built by volunteers over the years since 1991.


\section*{What We Do}

% This section might benefit from being expanded on after the bullets, but probably
% bullets first.

\begin{itemize} \large
    % Line spacing probably needs adjustment.
    % Describe other things as well
\item Help beginning users get installed and set up on Linux.
\item Advice and support for beginning users.
\item Tutorials for developer tools
\end{itemize}

\section*{Events}

\begin{tabular*}{0.9\textwidth}{p{2.5cm}cp{8cm}}
    InstallFest            & \tbd & let us help you install Linux on your laptop \\
    Lectures and workshops & \tbd & free and open source software-related talks by
        members and guests \\
    Open air open problems & \tbd & brief talks on open problems hosted in open
        spaces.
\end{tabular*}


%%%%%%%% END  1 %%%%%%%%
\vfill
\phantom{x}
\columnbreak{}
%%%%%%%% PAGE 2 %%%%%%%%

%%%%%%%%% OLD %%%%%%%%%
% Some of the stuff here can probably be used in the other sections, but we want to keep
% it fairly terse. Try not to get too deep on the FOSS philosophy; I'd probably avoid
% mentioning it at all outside of saying that this stuff is free.

% And if we do want to elaborate on things, we can do that on the back. The front is for a
% quick overview.

We are students who use or want to use Linux, the free and open source computer
operating system. Our members range in skill-level from deeply knowledgeable
gurus to the totally uninitiated, and we welcome everyone, whether or not you
have already stepped into the world of free and open source software (\foss).

Some of our members meet each week to chat, have lunch, and enjoy each others'
company while discussing Linux, and the ecosystem of software surrounding it.
While we are there, we often help each other troubleshoot problems, seek out
software to use, and help new users learn about and to install Linux on their
laptops.

\phantom{x}
\vfill
\begin{center} \tiny
    % Discreetly include the year, so we know how old printed our copies of this
    % flier are.
    \textcopyleft{} Copyleft \the\year{} UMBC Linux Users' Group \\
    No rights reserved. \\
    Git Version \input{git-describe}
\end{center}
%%%%%%%% END  2 %%%%%%%%
\end{multicols*}
\end{document}
