\documentclass[11pt]{article}

\usepackage[margin=0.25in,landscape]{geometry}
\usepackage{multicol}
\usepackage{graphicx} % for including graphics
\usepackage{textcomp} % for copyleft symbol
\usepackage{tabulary} % for fixed-width tables
\usepackage{booktabs} % for fancy horizontal rules
\usepackage{wrapfig}  % for small QR codes

\newcommand{\foss}{FOSS}
\newcommand{\tbd}{TBD}

\setlength{\columnsep}{0.5in}
\def\columnseprule{0.5pt}

\begin{document}
\begin{multicols*}{2}
%%%%%%%% PAGE 1 %%%%%%%%

\begin{center} \scshape \Huge
    %\phantom{x}
    Linux Users' Group
%    \vspace{-1em}
\end{center}

\begin{center} \Large
    Meetings
    Every Wednesday, Noon to 1 p.m. in ITE 234
\end{center}

%\vspace{-3em}

\section*{About Us}

We are students who use or want to use Linux. We meet weekly to chat and help
each other solve problems, and organize other events to promote free and open
source software.

What is Linux? Linux is an operating system similar to OS X or Windows, except
that it is free and open source, built by volunteers over the years since 1991.

We also help beginners through the process of installing and learning Linux,
provide tutorials, and act as the go-to resource for Linux questions and
answers.

\section*{Why Linux?}

\subsubsection*{Linux is free.}

%Both the operating system and most of the software we use on it are developed by
%volunteers and have their source code posted online. Each Linux distribution has
%thousands of useful programs available in online repositories ready to be
%downloaded and installed for free.

\subsubsection*{Linux is customizable.}

%Linux users can choose what software to install to customize their distribution
%to their liking. This includes everything from graphical elements, like how
%windows look, to core system configuration, like which background processes are
%started at boot.

\subsubsection*{Linux runs everywhere.}

%Because it is free, and highly customizable, Linux is popular for a huge variety
%of use cases. Web servers, mobile divices, and integrated systems all frequently
%run Linux. Learning to use it (even if you don't end up keeping it as your
%desktop operating system, which we hope you do) is a very useful job skill for
%any programmer.

%\end{itemize}

%%%%%%%% END  1 %%%%%%%%
\vfill
\phantom{x}
\columnbreak{}
%%%%%%%% PAGE 2 %%%%%%%%

\renewcommand{\arraystretch}{1.8}
\begin{tabular*}{\hsize}{>{\raggedright}p{0.23\hsize} >{\centering}p{0.1\hsize} p{0.55\hsize}}
    \multicolumn{3}{c}{\LARGE Fall 2015 Schedule of Events} \\
    \toprule \\

    Next Meeting & Sept~9 & Bring lunch and come chat with us about Linux
        and FOSS. \\

    Flavors of Linux & TODO & Learn about Linux and the properties of different
        Linux distributions \\

    VPS for Fun and Profit & Sept~30 & David Lachut on what he does with his
    personal servers \\

    Introduction to Puppet & Oct~1 & Nick Miller talks about the basics of Puppet \\

    Linux Awareness Week & \mbox{Oct~9-16} & Celebrate and be aware of
        Linux in your life in the week leading up to Installfest. \\

    InstallFest & Oct~16 & Let us help you try Linux, or install it
        for you. \\

    Inside Git & TBD & Alexander Bauer on Git and its internals \\

\end{tabular*}

\phantom{x}
\vfill
\begin{center} \tiny
    % Discreetly include the year, so we know how old printed our copies of this
    % flier are.
    \textcopyleft{} Copyleft \the\year{} UMBC Linux Users' Group \\
    No rights reserved. \\
    Git Version \input{git-describe}
\end{center}
%%%%%%%% END  2 %%%%%%%%
\end{multicols*}
\end{document}
